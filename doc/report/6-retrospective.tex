\chapter{Retrospective}

\section{Development description in detail}

\subsection{Sprint 0}
\subsubsection{Report}
Product backlog was defined during first meetings. Each item in the backlog was associated to an estimate of his development effort.
\\
The organizational structure of the project was defined too. We discussed tools and development environments to support continuous integration.
\\
Initial analysis and design process were performed in order to define main entities of the system and their interactions.
Therefore, it was decided to internally use an approach base on actors, which however would have been transparent to the final user (developer). This is due to an elevated complexity of the system, and an actor model could simplify few aspects. 
We decided to rely upon ``akka-core library'' and ``akka-http library''. 
\\
In the end, it was decided to carry out client and server modules development in parallel in order to have a working core in the shortest time. This core would be extended incrementally during the project lifetime.




\subsection{Sprint 1}
\subsubsection{Report}
The goal of this first sprint was to define main client and server interfaces and to set up a minimal working system that allows:
\begin{itemize}
	\item Game server execution 
	\item Room types definition
	\item Handling of client requests to create rooms.
\end{itemize}
These goals were completed successfully. 
\\
An issue regarding the extension of an already existing server was found due to internal ``Akka'' policies. 
This ended up in a revision on some decisions that were made about server creation. The result is that it is possible for the user to specify external routes which will be handled by the game server (if they don't conflict with the already existing ones internally used  by the game server itself). 
However, it is not possible to attach the game server to an existing one.
\\
However, even if in a different way than the one prospected at the beginning, the requirement can be considered fulfilled.   




\subsubsection{Retrospective}
The task effort concerning the creation of a game server based on an existing one was underestimated, and it created few troubles during the whole week.
\\
The team was able to complete all tasks and effectively use continuous integration and project management tools decided during first meetings.
\\
A problem that arose was that some ``scalastyle'' rules were too strict, and tests were failing for no good reasons. We decided then to disable such rules for further development.


\subsection{Sprint 2}
\subsubsection{Report}

During this week some of the main features of the library about clients and rooms have been developed.
\\
Tasks regarding web socket support was underestimated, and it was not fully completed during this sprint.
The interaction between clients and rooms required the definition of an ad hoc protocol to manage messages internally used by the library. This introduced some dependencies between client and server tasks causing little slowdown in the development process.
\\
Given the amount of work spent on such task, some others were left behind and not fully achieved too (e.g. the creation of rooms with custom options).
\\
Uncompleted tasks will be completed in the next sprint.



\subsubsection{Retrospective}
Even if the amount of work spent on this sprint was certainly greater than the previous one, the fact that the team was not able to complete all tasks requires some further considerations.
\\
It's important to better define the work in advance. For example, we should consider splitting tricky tasks in multiple sub-tasks when needed, so that hidden dependencies are found and parallel work can be easily planned and carried out.
\\
The creation of rooms with custom options was not completed due to the lack of communication about who was the volunteer of such task, and, in the end, we run out of time.
\\
Furthermore, team communication should be increased during the week.
This could help by solving in advance problems that arise during development.

\subsection{Sprint 3}
\subsubsection{Report}

In this week, pending tasks from previous sprint were completed.

While implementing automatic state update, we decided to use binary serialization instead of Json one. The latter required the user to define his own parsers while binary serialization allowed a more generic and simple approach.

Since the work done in the previous weeks contained the core features of the library, we were able to implement a first simple example of online multiplayer game: ``Rock-Paper-Scissor''. 
The main purpose was to verify effective usability of features produced so far, both for client and server sides.
The final code is also a good explanation about how to use the library and its basic functionalities.


\subsubsection{Retrospective}
Tasks were all completed this time; anyway, the lack of documentation and diagrams relative to some common concepts led to few integration issues on code produced by different team members. Sometimes, ad hoc refactoring was needed to keep high software quality throughout the development.
\\
Increased care must be granted to future documentation process.
\\
During the week, few unpredictable tests were added to the test suite.
As they were found, they were immediately fixed within the end of the sprint.




\subsection{Sprint 4}
\subsubsection{Report}
In this sprint, other main features were added, such as game loop on server rooms and client re-connection.
\\
Moreover, it was implemented ``MoneyGrabber'', a second game that takes advantage of more advanced features of the library, such as the automatic state update and the management of rooms with multiple players.
Therefore, it was possible to verify the actual versatility of the library showing a further and more complex example of use in a real context.
\\
All tasks were completed ahead of time.


\subsubsection{Retrospective}

During this sprint the team worked very well, probably helped by the solid core already implemented that was easily extensible.
\\
Probably, we should have considered the addition of some more tasks to the backlog in the sprint planning.
\\
By the way, the remaining time of the sprint was used by the whole team to improve the existing code quality and to produce some more detailed documentation.
\\
The only drawback was the underrate of game loop consequences on the whole system. Indeed, game loop, and also the already implemented state synchronization feature, might imperil system safety due to possible concurrency problems. A meeting with the team was required to improve some design aspects so it was possible to solve any issue while keeping it transparent to final user; it took a non negligible amount of time.

\subsection{Sprint 5}

\subsubsection{Report}
Matchmaking functionality was implemented during this sprint.
A revise on client-server communication protocol was required to integrate matchmaking service in the system.

\subsubsection{Retrospective}

Given the importance of the new matchmaking feature, we decided to develop a further example showing its use in an example application. This will be done in the next sprint.
\\
The revision of communication protocol was due to some non trivial matchmaking aspects that were non fully considered in advance, such as data flows and separation between normal and matchmaking rooms. 
\\
In the end, we noticed that client-side room closure feature was not explicitly defined yet. 
After a brief discussion, we decided that a developer could implement such functionality by using existing features (e.g auto-close feature on server-side) in a safer and robust manner. Hence, we considered that requirement unnecessary, and so the task as fulfilled.
\\
Thus, all requirements are satisfied, and library is almost ready for deployment.


\subsection{Sprint 6}

\subsubsection{Report}

During this sprint the team worked on the deployment process. Some refactoring was done on both code and package structure. Particular attention was given to visibility rules aspects.
The three usage examples have been refined and executable jars have been created for each. Every example comes with two jars, one for the client side application and one for the server side initialization.
\\
We were not able to complete the library publishing on Maven because the whole process require some more understanding.
In the next few days the team will work together to complete the task.



\subsubsection{Retrospective}


The complexity of the publishing process on Maven was underestimated.
This is an important requirement that existed from the beginning of the project, probably we should have discuss this topic in earliest Sprints. 
An initial discussion would have led to greater awareness, making it possible to increase the knowledge on the topic incrementally during the project lifetime.
\\
Also the basic package structure could have been settled in advanced. 
Indeed some (avoidable) extra-work was done to fix the structure both in source and test code.

\section{Final comments}
\subsection{Stefano Salvatori}
\subsection{Gabriele Guerrini}
\subsection{Riccardo Salvatori}