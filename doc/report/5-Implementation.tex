\chapter{Implementation}
\section{Data Serialization}
As decribed in design section \ref{sec:communication_design}, there are two main types of serialization we dealt with, that is:
\begin{itemize}
	\item Json serialization: request-response interaction
	\item Binary serialization: data exhanged through websockets
\end{itemize}

Regarding the first one, we decided to use the \texttt{spray-json} library since it is the recommended one when using akka.

About binary serialization, we chose to use \texttt{java.io} serialization. This is because there are no strict performance constraints in requirements, and we preferred to favor user-side usability: indeed, the user of the library is able to let exchange messages between clients and server just defining classes that extend \texttt{java.io.Serializable} interface.



\section{Stefano Salvatori}
In the early stages of the project I have mainly worked on the server package. First I defined the \texttt{GameServer} trait providing a basic implementation of it with a \texttt{ServerActor} that allowed to run a server listening on specific host an port; then I started working on the \texttt{ServerRoom} class and the \texttt{RoomActor} so that it would be possible for clients to connect to a room and for developers to create their own type of room. Concerning ServerRoom class i've also implemented the automatic close functionality and the periodic state synchronization to clients.

Later on I worked on the communication protocol implementing all the classes in the \texttt{common.communication} package. Regarding communication between client and server i've also designed and implemented the websocket utilities in the \texttt{server.communication} package. Here i had to use some advanced features of akka reactive streams. 

During the third and fourth sprint of the project i developed the \textit{MoneyGrabber} game as an example of the main functionalities of the library. 

\section{Gabriele Guerrini}

\section{Riccardo Salvatori}

Initially I worked on the Route service component. I used a DSL provided by the akka-http library to easily define and handle specific routes used by the library. Then I defined the \textit{RoomHandler} component that is in fact used by the RouteService to .
I worked on the client side api defining the client  room trait \textit{ClientRoom} with the implementations \textit{JoinableRoom} and \textit{JoinedRoom}. Therefore I worked on functionality to join, reconnect and leave rooms.
Furthermore I worked on the implementation of the \textit{matchmaking} functionality client side: \textit{ClientMatchmaker} and \textit{ClientMatchmakerActor}.

I implemented the \textit{RockPaperScissor} example using scalafx and prolog.

In the client side developing I was able to make an extensive use of reactive and asynchronus programming by the use of actors, futures and reactive streams.
I could appreciate the use of \textit{for comprehension} syntax provided by scala to combine a work with futures.
I was also  to improve my knowledge of  Akka library and actor programming; also combined with a functional approach.


\begin{itemize}

\item[\em{Client infrastructure}] Realization of the main client infrastructure based upon main concepts of \texttt{CoreClient} (i.e. the main actor of a client), client facade (\texttt{Client}) and \texttt{HttpService}. My version, realized in the first sprint, has been then improved in the second one by Riccardo Salvatori.

\item[\em{Room}] Design of rooms (\texttt{Room}, \texttt{BasicRoom}, \texttt{SharedRoom}, all client rooms, \texttt{ServerRoom}) and their interactions. Implementation of common rooms with contribution of Riccardo Salvatori.

\item[\em{Server room}] Design and implementation of \texttt{ServerRoom} featuring with Stefano Salvatori. In particular, I've realized the handling of room properties, game loop (\texttt{GameLoop}), and various other features contained in \texttt{ServerRoomBasicFeatures} such as private state, locking and matchmaking integration.

\item[\em{Room property}] Design of room property (\texttt{RoomProperty}) and its integration in client and server rooms both.

\item[\em{Room filters}] Design of room filters (\texttt{RoomFilters}) and its usage on client and server side both.

\item[\em{Json serialization}] Implementation of Json serialization (\texttt{RoomJsonSupport}).

\item[\em{Matchmaking}] Design of \texttt{Matchmaker}, implementation of the default one, handling of rooms with matchmaking (opposed to rooms without it) in \texttt{RoomHandler}.

\item[\em{Roll The Dice}] Implementation of the third example game \textit{``Roll The Dice''} that shows usage of matchmaking as well as other functionalities exhibited in the first two.

\end{itemize}

Since usability is a core aspect of the library, advanced features like reflection and annotations have been combined and widely used on \texttt{ServerRoom} in order to mantain the best of it while creating custom rooms. This way, the developer can freely define a room property just by declaring a field marked with an appropriate annotation.
\\
In the end, always talking about library usability, I developed a DSL to easily manage room properties and filters.





