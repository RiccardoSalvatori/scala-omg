\chapter{Sommario}
\markboth{SOMMARIO}{SOMMARIO}
Seguendo un percorso suddivisibile essenzialmente in due macro-fasi, la tesi si sviluppa attraverso otto capitoli. La prima fase, dedicata all'approfondimento di tematiche di ricerca ed alla ricognizione sullo stato dell'arte, è descritta attraverso i primi quattro capitoli. Gli ultimi quattro capitoli, invece, descrivono la successiva fase relativa all'analisi, alla progettazione e allo sviluppo di un prototipo nell'ambito di un ben preciso caso di studio. 

\paragraph{Capitolo 1}
Il primo capitolo affronta brevemente il rapporto tra le tecnologie informatiche (ICT) e l'healthcare. In particolare, viene illustrato il ruolo delle moderne tecnologie nell'ambito della cura delle persone con particolare riferimento ai sistemi a supporto dei soccorritori impiegati in situazioni d'emergenza.

\paragraph{Capitolo 2}
Il secondo capitolo esplora e descrive le caratteristiche dei sistemi di Pervasive Mobile Computing e analizza le problematiche collegate alla loro progettazione. Inoltre, sono affrontate due tematiche strettamente correlate a tale tipologica sistemi, ovvero Context-Awareness e Human-Computer Interaction.

\paragraph{Capitolo 3}
Nel terzo capitolo si entra nel dettaglio dei sistemi wearable a supporto delle tecniche di realtà aumentata. Dopo una breve introduzione ai concetti principali in materia di realtà aumentata, sono descritte le caratteristiche dei cosiddetti AR-Glasses corredate da un'analisi circa l'impiego di tale supporto wearable in scenari d'emergenza e soccorso in ambito healthcare.

\paragraph{Capitolo 4}
L'ultimo capitolo della prima fase della tesi, considera il sistema operativo Android ed in particolare, dopo una breve overview, descrive le caratteristiche d'interesse per poter utilizzare Android come tecnologia di riferimento per lo sviluppo di sistemi di Pervasive Mobile Computing. Tralasciando gli aspetti marginali e soprattutto quelli relativi alla progettazione di interfacce grafiche, il focus della trattazione è relativo ad uno studio approfondito dei meccanismi implementati in Android per favorire la progettazione di sistemi robusti, manutenibili e scalabili.

\paragraph{Capitolo 5}
La seconda fase del percorso di tesi si apre con la definizione del caso di studio. In particolare, in questo quinto capitolo si definisce un glossario per la descrizione dei termini di riferimento e una lista di scenari d'utilizzo del sistema.

\paragraph{Capitolo 6}
Il sesto capitolo affronta la fase di analisi dei requisiti relativamente alla porzione d'interesse del caso di studio. In particolare, quel che emerge sono le caratteristiche funzionali e non che il sistema oggetto della progettazione dovrà implementare. Inoltre, una sezione è dedicata all'identificazione delle caratteristiche a forte impatto innovativo che il sistema finale offrirà all'utilizzatore. 

\paragraph{Capitolo 7}
Il settimo capitolo affronta la progettazione del sistema descritto nel caso di studio e precedentemente analizzato. Partendo dalla definizione di un'architettura logica ispirata ai concetti presenti nel paradigma ad agenti, tutti i componenti vengono descritti dettagliatamente, corredando (se necessario) ogni fase di progettazione con alcuni riferimenti tecnologici. 

\paragraph{Capitolo 8}
L'ultimo capitolo della tesi riporta le strategie adottate per lo sviluppo prototipale delle caratteristiche più salienti e a maggior impatto innovativo del sistema progettato secondo quando descritto nel capitolo precedente. 